\documentclass[autodetect-engine,dvi=dvipdfmx,ja=standard,
               a4j,11pt]{bxjsarticle}

% 2019.6.19 14:00 hara

% 上記の説明ページにおける,レポート”例”について,一部訂正をします.
% 358行目に以下の1行を追加する.

% なお,下記の実行例の行頭に書かれた「\verb|%|」は,動作確認をしたRed hat Linux 3.2.2-5におけるターミナルのプロンプトである.

%%======== プレアンブル ============================================%%
\RequirePackage{geometry}
\geometry{reset,a4paper}
\geometry{hmargin=25truemm,top=25truemm,bottom=25truemm,footskip=10truemm}
\geometry{showframe} % はみ出し確認用

\usepackage{graphicx}

\usepackage{fancyvrb}
% \renewcommand{\theFancyVerbLine}{\texttt{\footnotesize{\arabic{FancyVerbLine}:}}}

%%======== レポートタイトル等 ======================================%%
\title{プログラミング演習1 \\
       期末レポート}

\author{氏名: 池田 海斗 (IKEDA, Kaito) \\
        学生番号: 09501502}

\date{出題日: 2020年04月22日 \\
      提出日: 2020年06月08日 \\
      締切日: 2020年06月10日 \\}

%%======== 本文 ====================================================%%
\begin{document}
\maketitle

%%%%%%%%%%%%%%%%%%%%%%%%%%%%%%%%%%%%%%%%%%%%%%%%%%%%%%%%%%%%%%%%%%%%%%
\section{概要} \label{sec:1}
%%%%%%%%%%%%%%%%%%%%%%%%%%%%%%%%%%%%%%%%%%%%%%%%%%%%%%%%%%%%%%%%%%%%%%

本演習では,C言語を用いて名簿管理プログラムの制作を行う.
このプログラムではコマンド入力によるデータの読み込み・書き出しを行ったり,整列・検索を行うことができる.
また,標準入力からコンマで区切られた文字列を入力しデータの登録を行うことができる.

本レポートでは,演習中に取り組んだ課題として,
以下の課題1から課題6についての内容を報告する.

\begin{description}
  \item[課題1] 文字列操作の基礎:\verb|subst|関数と\verb|split|関数の実装
  \item[課題2] 構造体や配列を用いた名簿データの定義
  \item[課題3] 標準入力の取得と構文解析
  \item[課題4] CSVデータ登録処理の実装
  \item[課題5] コマンド中継処理の実装
  \item[課題6] コマンドの実装:\verb|%P|コマンド
\end{description}

また,取り組んだ課題のうち,以下の課題については詳細な考察を行った.

\begin{description}
  \item[課題1] 文字列操作の基礎:\verb|subst|関数と\verb|split|関数の実装
  \item[課題3] 標準入力の取得と構文解析
  \item[課題6] コマンドの実装:\verb|%P|コマンド
\end{description}


%%%%%%%%%%%%%%%%%%%%%%%%%%%%%%%%%%%%%%%%%%%%%%%%%%%%%%%%%%%%%%%%%%%%%%
\section{プログラムの作成方針} \label{sec:2}
%%%%%%%%%%%%%%%%%%%%%%%%%%%%%%%%%%%%%%%%%%%%%%%%%%%%%%%%%%%%%%%%%%%%%%

本演習で作成したプログラムが満たすべき要件と仕様として,
「(1) 基本要件」と「(2) 基本仕様」を示す.

\subsubsection*{(1) 基本要件}

\begin{enumerate}
  \setlength{\parskip}{0em} \setlength{\itemsep}{0.25em}
    \item プログラムは,その実行中,少なくとも10,000件の名簿データをメモリ中に保持できるようにすること.
    \item 名簿データは,「ID, 氏名, 誕生日, 住所, 備考」を,1つのデータとして扱えるようにすること.
    \item プログラムとしての動作や名簿データの管理のために,以下の機能を持つコマンドを実装すること.
    \begin{enumerate} \setlength{\parskip}{0em} \setlength{\itemsep}{0.25em}
        \item プログラムの正常な終了
        \item 登録された名簿データのデータ数表示
        \item 名簿データの全数表示,および,部分表示
        % \item 名簿データのファイルへの保存,および,ファイルからの復元
        % \item 名簿データの検索と表示
        % \item 名簿データの整列
    \end{enumerate}
    \item 標準入力からのユーザ入力を通して,データ登録やデータ管理等の操作を可能とすること.
    \item 標準出力には,コマンドの実行結果のみを出力すること.
\end{enumerate}

\subsubsection*{(2) 基本仕様}

\begin{enumerate}
  \setlength{\parskip}{0em} \setlength{\itemsep}{0.25em}
    \item 名簿データは,コンマ区切りの文字列(\textbf{CSV入力}と呼ぶ)で表されるものとし,
          図\ref{fig:csvdata}に示したようなテキストデータを処理できるようにする.
    \item コマンドは,\%で始まる文字列(\textbf{コマンド入力}と呼ぶ)とし,
          表\ref{tab:commands}にあげたコマンドをすべて実装する.
    \item 1つの名簿データは,C言語の構造体 (\texttt{struct}) を用いて,
          構造を持ったデータとしてプログラム中に定義し,利用する.
    \item 全名簿データは,``何らかのデータ構造''を用いて,メモリ中に保持できるようにする.
    \item コマンドの実行結果以外の出力は,標準エラー出力に出力する.
\end{enumerate}

%========================= EXAMPLE CSV ================================%
\begin{figure}[b]
\centering
\begin{Verbatim}[frame=single, xleftmargin=10mm, xrightmargin=5mm, gobble=4,
                 fontsize=\small, numbers=left, firstnumber=1]
    5100046,The Bridge,1845-11-2,14 Seafield Road Longman Inverness,SEN Unit 2.0 Open
    5100127,Bower Primary School,1908-1-19,Bowermadden Bower Caithness,01955 641225 ...
    5100224,Canisbay Primary School,1928-7-5,Canisbay Wick,01955 611337 Primary 56 3...
    5100321,Castletown Primary School,1913-11-4,Castletown Thurso,01847 821256 01847...
\end{Verbatim}
    \caption{名簿データのCSV入力形式の例.1行におさまらないデータは...で省略した.}
    \label{fig:csvdata}
\end{figure}
%========================= EXAMPLE CSV ================================%

%========================= COMMAND LIST ================================%
\begin{table}[b]
\centering
    \caption{実装するコマンド}
    \label{tab:commands}
    \begin{tabular}{|l|ll|l|}
    \hline
    \textbf{コマンド} & \textbf{意味} & & \textbf{備考} \\
    \hline\hline
    \verb|%Q| & 終了 & (Quit) & \\
    \hline
    \verb|%C| & 登録内容のチェック & (Check) & 1行目に登録数を必ず表示 \\
    \hline
    \verb|%P n| & 先頭から$n$件表示 & (Print) & $n$が$0$ $\to$ 全件表示, \\
                & & & $n$が負 $\to$ 後ろから$-n$件表示 \\
    \hline
    % \verb|%R file| & $file$から読込み & (Read) & \\
    % \hline
    % \verb|%W file| & $file$への書出し & (Write) & \\
    % \hline
    % \verb|%F word| & 検索結果を表示 & (Find) & \verb|%P|と同じ形式で表示 \\
    % \hline
    % \verb|%S n| & $CSV$の$n$番目の項目で整列 & (Sort) & 表示はしない \\
    % \hline
    \end{tabular}
\end{table}
%========================= COMMAND LIST ================================%


%%%%%%%%%%%%%%%%%%%%%%%%%%%%%%%%%%%%%%%%%%%%%%%%%%%%%%%%%%%%%%%%%%%%%%
\section{プログラムの説明} \label{sec:3}
%%%%%%%%%%%%%%%%%%%%%%%%%%%%%%%%%%%%%%%%%%%%%%%%%%%%%%%%%%%%%%%%%%%%%%

プログラムリストは\ref{sec:7}章に添付している.
その中でも,最終的なプログラムとして\ref{code:6}節を参照しながら説明を行なっていく.
プログラムは全部で232行からなる.
以下では,\ref{sec:1}節の課題ごとに,プログラムの主な構造について説明する.

%--------------------------------------------------------------------%
\subsection{文字列操作の基礎:\texttt{subst}関数と\texttt{split}関数の実装}
%--------------------------------------------------------------------%

まず,汎用的な文字列操作関数として,
\verb|subst()|関数を137--153行目で宣言し,
\verb|split()|関数を155--173行目で宣言している.
また,これらの関数で利用するために,\verb|stdio.h|,\verb|string.h|のヘッダファイルを読み込んでいる.

\verb|subst(STR, C1, C2)|関数は,
\verb|STR|が指す文字列中の,文字\verb|C1|を文字\verb|C2|に置き換える.
プログラム中では,\verb|get_line()|関数の中で,
文字``\verb|\n|''を終端ヌル文字``\verb|\0|''置き換えるために,この関数を呼び出している.
この関数では,文字\verb|C1|を文字\verb|C2|に置き換えた回数を返り値とする.

\verb|split(STR, RET, SEP, MAX)|関数では,
\verb|STR|が指す文字列を,最大\verb|MAX|個まで文字\verb|SEP|の箇所で区切り,配列\verb|RET|に格納していく.
プログラム中では,\verb|get_line()|関数で取得した1行を,カンマで区切りで配列に格納する際に呼び出される.
また,ハイフン区切りで入力される生年月日のデータを,年・月・日に分けて配列に格納する際にも用いる.
この関数では,配列に分割された個数を返り値とする.

%--------------------------------------------------------------------%
\subsection{構造体や配列を用いた名簿データの定義}
%--------------------------------------------------------------------%

本プログラムでは,構造体の配列に名簿データを格納していく.
7--12行目で,\verb|date|構造体を定義し,
14--21行目で,\verb|profile|構造体を定義している.
ID, 氏名, 誕生日, 住所, 備考の5つの組み合わせで1つの名簿データとなる.
名簿データの誕生日の箇所にあたっては\verb|date|構造体を用い,年・月・日を\verb|int|型として別々に保存しておく.
こうすることで,並び替えなどを行う際に効率よく実行ができる.

そして,24行目の\verb|profile_data_store|変数で全名簿データを管理し,
23行目の\verb|profile_data_nitems|変数で,名簿データの個数を管理している.

%--------------------------------------------------------------------%
\subsection{標準入力の取得と構文解析}
%--------------------------------------------------------------------%

\verb|get_line(LINE)|関数は,
標準入力で読み込まれた名簿データを1行ずつ取り出し,\verb|LINE|に代入する.
読み込む行がない場合,文字列が1024文字を超えてバッファオーバーランをする場合,
ユーザが改行コードを入力した場合に返り値\verb|0|を返す.

\verb|parse_line(LINE)|関数では,
\verb|get_line(LINE)|関数で読み込まれた1行が,コマンド入力かデータ入力であるか条件分岐を行う.
コマンドの場合,コマンド名と引数に分けてその値を\verb|exec_command(CMD, PARAM)|関数に引き渡し,
入力データであった場合は\verb|new_profile(PROFILE_DATA_STORE, LINE)|関数で登録を行う.

%--------------------------------------------------------------------%
\subsection{CSVデータ登録処理の実装}
%--------------------------------------------------------------------%

\verb|new_profile(PROFILE_DATA_STORE, LINE)|関数内では,
グローバル変数\verb|profile_data_store|の任意の配列番号のポインタを受け取る.
\verb|split|関数を用いて1行をカンマで5つに区切り,その中の3番目の要素を更にハイフンで3つに区切る.
そして,それぞれの要素をID・名前・誕生日・住所・備考として,
配列\verb|profile_data_store|の\verb|profile_data_nitems|番目の要素としてデータを保存する.
誕生日に関しても同様,構造体\verb|date|の型で年・月・日に細分化しデータを格納する.
また,並び替えを行いやすいように,ID・誕生日に関しては\verb|int|型に変換して保存する.

%--------------------------------------------------------------------%
\subsection{コマンド中継処理の実装}
%--------------------------------------------------------------------%

\verb|exec_command(CMD, PARAM)|関数では,
コマンド文字と引数を受け取りそれによって関数を呼び出す,
案内所のような役割を行っている.
引数はポインタで取得するようにしてある.
また一致するコマンドが見つからない場合,処理は行わないようにしてある.

%--------------------------------------------------------------------%
\subsection{コマンドの実装:\texttt{\%P}コマンド}
%--------------------------------------------------------------------%

\verb|cmd_print(CMD, PARAM)|関数内では,
入力された引数が0, 正, 負であるかによって条件分岐される.
また,登録されている要素数の絶対値より大きい引数が入力された場合,
要素数分の処理のみ実行する.
また,出力のフォーマットは\verb|db_sample|と同じように揃えている.
引数が不正な値である場合でも,\verb|atoi()|関数を用いることで
$0$を入力したのと同様の動作を行うようにしてある.
これは\verb|db_sample|と同じ仕様である.


%%%%%%%%%%%%%%%%%%%%%%%%%%%%%%%%%%%%%%%%%%%%%%%%%%%%%%%%%%%%%%%%%%%%%%
\section{プログラムの使用法と実行結果} \label{sec:4}
%%%%%%%%%%%%%%%%%%%%%%%%%%%%%%%%%%%%%%%%%%%%%%%%%%%%%%%%%%%%%%%%%%%%%%

%--------------------------------------------------------------------%
\subsection{プログラムの概要}
%--------------------------------------------------------------------%

本プログラムは名簿データを管理するためのプログラムである.
コマンドを入力する際には,\%で始まる英字を標準入力に打ち込み,引数で詳細な指示を行う.
コマンドの詳細は\ref{sec:2}節に記述してある.
また標準入力からのデータ入力も可能としている.

%--------------------------------------------------------------------%
\subsection{実行環境}
%--------------------------------------------------------------------%

プログラムは,MacOS Catalina 10.15.4 で動作を確認しているが,
一般的な UNIX で動作することを意図している.
なお,以降の実行例における,行頭の\verb|%|記号は,
MacOS Catalina 10.15.4におけるターミナルのプロンプトである.

%--------------------------------------------------------------------%
\subsection{コンパイル方法}
%--------------------------------------------------------------------%

まず,\verb|gcc|でコンパイルすることで,プログラムの実行ファイルを生成する.
ここで,\verb|-Wall|とは警告オプションを全て有効にするためのオプションであり,
\verb|-o|とは実行ファイルの名前を指定するオプションである.
これらのオプションをつけることで,コードの視認性を高めたり無駄なコードを省くことができ,
他のソースコードの実行ファイルとの識別が容易である.

{\fontsize{10pt}{11pt} \selectfont
 \begin{verbatim}
   % gcc -Wall -o eop_06_09501502 eop_06_09501502.c
 \end{verbatim}
}

%--------------------------------------------------------------------%
\subsection{実行方法}
%--------------------------------------------------------------------%

次に,プログラムを実行する.
以下の実行例は,プログラム実行中のデータの入力を模擬するため,
CSVファイルを標準入力により与えることで,実行する例を示している.
通常の利用においては,\verb|%R file|によりデータを読み込む.

{\fontsize{10pt}{11pt} \selectfont
 \begin{verbatim}
   $ ./eop_06_09501502.out < stdin.csv
 \end{verbatim}
}

%--------------------------------------------------------------------%
\subsection{出力結果}
%--------------------------------------------------------------------%

%---------------------------------------------------------------%
\subsubsection{第1回講義 (\texttt{subst}関数)}
%---------------------------------------------------------------%

今回は標準入力は使用しない.
第\ref{code:1}節に記述してあるプログラムを実行すると,以下のような出力が得られる.

{\fontsize{10pt}{11pt} \selectfont
 \begin{verbatim}
    文字列: ABCDE
    置換文字: A -> B

    実行結果: BBCDE
    差分: 1
 \end{verbatim}
}

今回のプログラムにはユーザによる入力はないので,出力結果について説明する.
1 $\cdot$ 2行目について,予め登録しておいた文字列\verb|ABCDE|の中から,
文字\verb|A|の文字を\verb|B|に置換するプログラムであることを示す.
3行目実行結果は置換後の文字列を表し,4行目には置換が行われた回数を表示している.

%---------------------------------------------------------------%
\subsubsection{第2回講義 (\texttt{split}関数と\texttt{get\_line}関数)}
%---------------------------------------------------------------%

第\ref{code:2}節に記述してあるプログラムを実行すると,
プログラムの出力結果としてCSVデータの各項目が読みやすい形式で出力される.
例えば,下記の \verb|cvsdata.csv| に対して,

{\fontsize{10pt}{11pt} \selectfont
 \begin{verbatim}
  5100046,The Bridge,1845-11-2,14 Seafield Road Longman Inverness,SEN Unit 2.0 Open
  5100127,Bower Primary School,1908-1-19,Bowermadden Bower Caithness,01955 641225 Pri...
  5100224,Canisbay Primary School,1928-7-5,Canisbay Wick,01955 611337 Primary 56 3.5 Open
  5100321,Castletown Primary School,1913-11-4,Castletown Thurso,01847 821256 01847 82...
  5100429,Crossroads Primary School,1893-2-24,Dunnet Thurso,01847 851629 01847 851629 ...
 \end{verbatim}
}

\noindent
以下のような出力が得られる.

{\fontsize{10pt}{11pt} \selectfont
 \begin{verbatim}
  Line1
  >ret[0] = '5100046'
  >ret[1] = 'The Bridge'
  >ret[2] = '1845-11-2'
  >ret[3] = '14 Seafield Road Longman Inverness'
  >ret[4] = 'SEN Unit 2.0 Open'
 Line2
  >ret[0] = '5100127'
  >ret[1] = 'Bower Primary School'
  >ret[2] = '1908-1-19'
  >ret[3] = 'Bowermadden Bower Caithness'
  >ret[4] = '01955 641225 Primary 25 2.6 Open'
 Line3
  >ret[0] = '5100224'
  >ret[1] = 'Canisbay Primary School'
  >ret[2] = '1928-7-5'
  >ret[3] = 'Canisbay Wick'
  >ret[4] = '01955 611337 Primary 56 3.5 Open'
 Line4
  >ret[0] = '5100321'
  >ret[1] = 'Castletown Primary School'
  >ret[2] = '1913-11-4'
  >ret[3] = 'Castletown Thurso'
  >ret[4] = '01847 821256 01847 821256 Primary 137 8.5 Open'
 Line5
  >ret[0] = '5100429'
  >ret[1] = 'Crossroads Primary School'
  >ret[2] = '1893-2-24'
  >ret[3] = 'Dunnet Thurso'
  >ret[4] = '01847 851629 01847 851629 Primary 29 2.4 Open'
 \end{verbatim}
}

まず,入力データについて説明する.
このCSVファイルには5件のデータを保存しており,
それぞれの項目はカンマで区切られている.

また出力結果について,\verb|Line*|はCSVファイルの行数を表し,
\verb|ret[*]|は配列の番号とその中に格納している文字列を表している.

%---------------------------------------------------------------%
\subsubsection{第3回講義 (\texttt{parse\_line}関数)}
%---------------------------------------------------------------%

第\ref{code:3}節に記述してあるプログラムを実行すると,
プログラムの出力結果としてCSVデータの各項目が読みやすい形式で出力される.
例えば,下記の \verb|cvsdata.csv| に対して,

{\fontsize{10pt}{11pt} \selectfont
 \begin{verbatim}
  5100046,The Bridge,1845-11-2,14 Seafield Road Longman Inverness,SEN Unit 2.0 Open
  %P contoso
  5100224,Canisbay Primary School,1928-7-5,Canisbay Wick,01955 611337 Primary 56 3.5 Open
  %C
  %Q
 \end{verbatim}
}

\noindent
以下のような出力が得られる.

{\fontsize{10pt}{11pt} \selectfont
 \begin{verbatim}
   >ret[0] = '5100046'
   >ret[1] = 'The Bridge'
   >ret[2] = '1845-11-2'
   >ret[3] = '14 Seafield Road Longman Inverness'
   >ret[4] = 'SEN Unit 2.0 Open'
  %P command is invoked with arg: 'contoso'
   >ret[0] = '5100224'
   >ret[1] = 'Canisbay Primary School'
   >ret[2] = '1928-7-5'
   >ret[3] = 'Canisbay Wick'
   >ret[4] = '01955 611337 Primary 56 3.5 Open'
  %C command is invoked with no arg
 \end{verbatim}
}

まず,入力データについて説明する.
コマンドは\verb|%P|, \verb|%C|, \verb|%Q|の3つを記述している.
また,このCSVファイルには2件のデータを保存しており,
それぞれの項目はカンマで区切られている.

また出力結果について,\verb|ret[*]|は配列の番号とその中に格納している文字列を表している.

%---------------------------------------------------------------%
\subsubsection{第4回講義 (\texttt{new\_profile}関数)}
%---------------------------------------------------------------%

第\ref{code:4}節に記述してあるプログラムを実行すると,
プログラムの出力結果としてCSVデータの各項目が読みやすい形式で出力される.
例えば,下記の \verb|stdin.csv| に対して,

{\fontsize{10pt}{11pt} \selectfont
 \begin{verbatim}
  5100046,The Bridge,1845-11-2,14 Seafield Road Longman Inverness,SEN Unit 2.0 Open
  %P contoso
  5100224,Canisbay Primary School,1928-7-5,Canisbay Wick,01955 611337 Primary 56 3.5 Open
  %C
  %Q
 \end{verbatim}
}

\noindent
以下のような出力が得られる.

{\fontsize{10pt}{11pt} \selectfont
 \begin{verbatim}
 %P command is invoked with arg: 'contoso'
 %C command is invoked with no arg
 \end{verbatim}
}

まず,入力データについて説明する.
コマンドは\verb|%P|, \verb|%C|, \verb|%Q|の3つを記述している.
また,このCSVファイルには2件のデータを保存しており,
それぞれの項目はカンマで区切られている.
\%Qコマンドはプログラムの終了を意味している.

また今回はデータの登録のみをバックグラウンドで行っている.
出力結果については,コマンド名とその引数を表示させている.

%---------------------------------------------------------------%
\subsubsection{第5回講義 (\texttt{\%C \%P}コマンド)}
%---------------------------------------------------------------%

第\ref{code:5}節に記述してあるプログラムを実行すると,
プログラムの出力結果としてCSVデータの各項目が読みやすい形式で出力される.
例えば,下記の \verb|stdin.csv| に対して,

{\fontsize{10pt}{11pt} \selectfont
 \begin{verbatim}
  5100046,The Bridge,1845-11-2,14 Seafield Road Longman Inverness,SEN Unit 2.0 Open
  5100224,Canisbay Primary School,1928-7-5,Canisbay Wick,01955 611337 Primary 56 3.5 Open
  %P 0
  %P 2
  %P -3
  %C
  %Q
 \end{verbatim}
}

\noindent
以下のような出力が得られる.

{\fontsize{10pt}{11pt} \selectfont
 \begin{verbatim}
  Id    : 5100046
  Name  : The Bridge
  Birth : 1845-11-02
  Addr. : 14 Seafield Road Longman Inverness
  Comm. : SEN Unit 2.0 Open

  Id    : 5100224
  Name  : Canisbay Primary School
  Birth : 1928-07-05
  Addr. : Canisbay Wick
  Comm. : 01955 611337 Primary 56 3.5 Open


  Id    : 5100046
  Name  : The Bridge
  Birth : 1845-11-02
  Addr. : 14 Seafield Road Longman Inverness
  Comm. : SEN Unit 2.0 Open

  Id    : 5100224
  Name  : Canisbay Primary School
  Birth : 1928-07-05
  Addr. : Canisbay Wick
  Comm. : 01955 611337 Primary 56 3.5 Open


  Id    : 5100046
  Name  : The Bridge
  Birth : 1845-11-02
  Addr. : 14 Seafield Road Longman Inverness
  Comm. : SEN Unit 2.0 Open

  Id    : 5100224
  Name  : Canisbay Primary School
  Birth : 1928-07-05
  Addr. : Canisbay Wick
  Comm. : 01955 611337 Primary 56 3.5 Open

  2 profile(s)
 \end{verbatim}
}

まず,入力データについて説明する.
コマンドは\verb|%P|, \verb|%C|, \verb|%Q|の3つを記述している.
また,このCSVファイルには2件のデータを保存しており,
それぞれの項目はカンマで区切られている.
出力結果については,コマンド名とその引数を表示させている.

\verb|%Q|コマンドはプログラムの終了を意味している.
\verb|%P|コマンドでは,引数が0の場合・引数が要素数と同じ場合・引数の絶対値が要素数よりも多い場合の
3通りを実行している.
また\verb|%C|コマンドを実行すると,要素数が表示される.

%---------------------------------------------------------------%
\subsubsection{第6回講義 (最終コード)}
%---------------------------------------------------------------%

第\ref{code:6}節に記述してあるプログラムを実行すると,
プログラムの出力結果としてCSVデータの各項目が読みやすい形式で出力される.
例えば,下記の \verb|stdin.csv| に対して,

{\fontsize{10pt}{11pt} \selectfont
 \begin{verbatim}
  5100046,The Bridge,1845-11-2,14 Seafield Road Longman Inverness,SEN Unit 2.0 Open
  5100224,Canisbay Primary School,1928-7-5,Canisbay Wick,01955 611337 Primary 56 3.5 Open
  ...
  8681040,St Francis Day Unit,1922-11-22,Cora Campus,01505 862899 Special 31 6.8 Open
  %C
  %P 0
  %P 2892
  %P -2893
  %Q
 \end{verbatim}
}

\noindent
以下のような出力が得られる.

{\fontsize{10pt}{11pt} \selectfont
 \begin{verbatim}
  2892 profile(s)

  Id    : 5100046
  Name  : The Bridge
  Birth : 1845-11-02
  Addr. : 14 Seafield Road Longman Inverness
  Comm. : SEN Unit 2.0 Open
  ...
  Id    : 8681040
  Name  : St Francis Day Unit
  Birth : 1922-11-22
  Addr. : Cora Campus
  Comm. : 01505 862899 Special 31 6.8 Open


  Id    : 5100046
  Name  : The Bridge
  Birth : 1845-11-02
  Addr. : 14 Seafield Road Longman Inverness
  Comm. : SEN Unit 2.0 Open
  ...
  Id    : 8681040
  Name  : St Francis Day Unit
  Birth : 1922-11-22
  Addr. : Cora Campus
  Comm. : 01505 862899 Special 31 6.8 Open


  Id    : 5100046
  Name  : The Bridge
  Birth : 1845-11-02
  Addr. : 14 Seafield Road Longman Inverness
  Comm. : SEN Unit 2.0 Open
  ...
  Id    : 8681040
  Name  : St Francis Day Unit
  Birth : 1922-11-22
  Addr. : Cora Campus
  Comm. : 01505 862899 Special 31 6.8 Open

 \end{verbatim}
}

まず,入力データについて説明する.
コマンドは\verb|%P|, \verb|%C|, \verb|%Q|の3つを記述している.
また,このCSVファイルには2892件のデータを保存しており,
それぞれの項目はカンマで区切られている.
要素数を増やした際でも正常に動くかテストを行なったため,一部データを...で省略してある.
また出力結果の整合性は,\verb|db-sample|の出力結果と,\verb|diff|コマンドを用いて確認してある.

\verb|%Q|コマンドはプログラムの終了を意味している.
\verb|%P|コマンドでは,引数が0の場合・引数が要素数と同じ場合・引数の絶対値が要素数よりも多い場合の
3通りを実行している.
また\verb|%C|コマンドを実行すると,要素数が表示される.


%%%%%%%%%%%%%%%%%%%%%%%%%%%%%%%%%%%%%%%%%%%%%%%%%%%%%%%%%%%%%%%%%%%%%%
\section{考察} \label{sec:5}
%%%%%%%%%%%%%%%%%%%%%%%%%%%%%%%%%%%%%%%%%%%%%%%%%%%%%%%%%%%%%%%%%%%%%%

\ref{sec:3}章のプログラムの説明,および,\ref{sec:4}章の使用法と実行結果から,
演習課題として作成したプログラムが,
\ref{sec:1}章で述べた基本要件と基本仕様のいずれも満たしていることを示した.
ここでは,個別の課題のうち,以下の3つの項目について,考察を述べる.

\begin{enumerate}
\setlength{\parskip}{2pt} \setlength{\itemsep}{2pt}
    \item 文字列操作の基礎:\verb|subst|関数と\verb|split|関数の実装
    \item 標準入力の取得と構文解析
    \item コマンドの実装:\verb|%P|コマンド
\end{enumerate}

%--------------------------------------------------------------------%
\subsection{「文字列操作の基礎:\texttt{subst}関数と\texttt{split}関数の実装」に関する考察}
%--------------------------------------------------------------------%

\subsubsection{「\texttt{subst}関数」に関する考察}

ここでは\verb|subst|関数について考察を行う.練習問題の解答例を見てみると,
\verb|main|関数内で複数の文字列を変換できるようにループさせており,
また配列に保存した複数の文字を置換した結果も出力できるようにしてある.
このことから,最終的に完成するプログラムでは,配列に保存された文字列を複数個
\verb|subst|関数にかけるのではないかと考察できる.

また,\verb|for|文で囲まれた部分に着目すると,
他の言語での\verb|foreach|文に似たような動作をしていることがわかった.

\subsubsection{「\texttt{split}関数」に関する考察}

次に,\verb|split|関数についての考察を行う.
今回工夫したところは,カンマを終端文字に置き換えることで,
文字列の先頭にポインタを合わせることで一気に文字列をコピーできることである.
その後,ポインタを終端ヌル文字の分1つ右へ移動させることで,次の文字列の文頭にポインタを移動している.


%--------------------------------------------------------------------%
\subsection{「標準入力の取得と構文解析」に関する考察}
%--------------------------------------------------------------------%

\subsubsection{「\texttt{get\_line}関数」に関する考察}

\verb|get_line|関数についての考察を行う.
今回重要なポイントとなってくるところは,\verb|fgets|の最大文字数を1024ではなく1025に設定したところである.
1024文字以上であることを確認するために,文字を1文字多く取得しそこに文字があるかで返り値を変えている.
しかし,\verb|strlen|関数は終端文字をカウントしないので,\verb|strlen|関数の値が1025文字以上を含まないようにしている.

\subsubsection{「\texttt{parse\_line}関数」に関する考察}

\verb|parse_line|関数についての考察を行う.
まずコマンドか否かの判別は,文字頭の値が\verb|%|で始まるかどうかで行っている.
これは,データ入力の場合文字列の始めは\verb|[0-9]|で始まるので,上記の条件に含まれることはないからである.
96行目では,ポインタの4文字目以降からヌル終端文字までを配列に格納している.


%--------------------------------------------------------------------%
\subsection{コマンドの実装:\texttt{\%P}コマンド}
%--------------------------------------------------------------------%

\verb|cmd_print(CMD, PARAM)|関数についての考察を行う.
まず,\verb|atoi()|関数で引数を\verb|int|型に変換する.
引数が0の場合,正の場合,負の場合で条件分岐を行う.
また,引数の絶対値が要素数よりも大きい場合は,要素数を用いて処理を行う.
今回考慮するべきは,引数が入力されていない場合や数字以外が入力された場合であり,
\verb|atoi()|関数ではそのような値に$0$という値を返すため
\verb|db_sample|の仕様通りになる.


%%%%%%%%%%%%%%%%%%%%%%%%%%%%%%%%%%%%%%%%%%%%%%%%%%%%%%%%%%%%%%%%%%%%%%
\section{感想} \label{sec:6}
%%%%%%%%%%%%%%%%%%%%%%%%%%%%%%%%%%%%%%%%%%%%%%%%%%%%%%%%%%%%%%%%%%%%%%

今回のプログラムでは,\verb|split()|関数で苦労した.
私は最初,カンマ区切りの要素数よりも\verb|max|値の方が小さければ,
配列\verb|ret|の\verb|max|番目の要素には模範解答のように
残りの文字列をむりやり入れるような仕様にはしていなかった.

\ref{code:7}節のコードのように,
受け取った文字列\verb|str|をまず初めに\verb|subst()|関数にかけることで,
カンマを全て終端\verb|NULL|文字に置き換え,ポインタを用いて一気に文字列を代入できる方法を思いついた.
カンマを終端\verb|NULL|文字に置き換えるという発想自体は良かったのだが,
これだと仕様を満たしていなかったためテストを通過できなかった.

また,C言語にもともと用意されている関数を使う場面も多く,
リファレンスから行いたい処理をしてくれる関数を見つけ出す検索力も
プログラミングには必要だなと感じた.
希望する動作を行ってくれる関数を探し出す力,バグの原因と解決策を自分で見つけ出す力など,
行く行くはプログラムを行なっていく際に自分で解決する力を養うことが今の僕たちには大切だなと感じた.


%%%%%%%%%%%%%%%%%%%%%%%%%%%%%%%%%%%%%%%%%%%%%%%%%%%%%%%%%%%%%%%%%%%%%%
\section{作成したプログラム} \label{sec:7}
%%%%%%%%%%%%%%%%%%%%%%%%%%%%%%%%%%%%%%%%%%%%%%%%%%%%%%%%%%%%%%%%%%%%%%

作成したプログラムを以下に添付する.
なお,\ref{sec:1}章に示した課題については,
\ref{sec:4}章で示したようにすべて正常に動作したことを付記しておく.

%---------------------------------------------------------------%
\subsection{第1回 \texttt{subst}関数} \label{code:1}
%---------------------------------------------------------------%

\begin{Verbatim}[numbers=left, xleftmargin=10mm, numbersep=6pt,
    fontsize=\small, baselinestretch=0.8]
#include <stdio.h>

int subst(char *str, char c1, char c2) {
  int diff = 0;
  char *p;

  p = str;

  while (*p != '\0') {
    if (*p == c1) {
      *p = c2;
      diff++;
    }
    p++;
  }

  return diff;
}

int main(void) {
  char c1 = 'A';
  char c2 = 'B';
  char s[80] = "ABCDE";

  printf("文字列: %s\n置換文字: %c -> %c\n\n", s, c1, c2);
  printf("実行結果: %s\n差分: %d\n", s, subst(s, c1, c2));
}

\end{Verbatim}

%---------------------------------------------------------------%
\subsection{第2回 \texttt{split}関数と\texttt{get\_line}関数} \label{code:2}
%---------------------------------------------------------------%

\begin{Verbatim}[numbers=left, xleftmargin=10mm, numbersep=6pt,
    fontsize=\small, baselinestretch=0.8]
#include <stdio.h>
#include <string.h>

int subst(char *str, char c1, char c2) {
    int diff = 0;
    char *p;

    p = str;
    while (*p != '\0') {
        if (*p == c1) {
            *p = c2;
            diff++;
        }
        p++;
    }
    return diff;
}

int split(char *str, char *ret[], char sep, int max) {
    int i, count = 0;
    for (i=0; i<max; i++) {
        ret[i] = str;
        str += strlen(str)+1;
        count++;
    }
    return count;
}

int get_line(char *line) {
    if (fgets(line, 1026, stdin) == NULL || strlen(line) >1024 || *line == '\n') {
        return 0;
    } else {
        subst(line, '\n', '\0');
        subst(line, ',', '\0');
        return 1;
    }
}

int main(void) {
    int i, count = 1, max = 5;
    char line[1024] = {0};
    char *ret[80] = {0};
    char sep = ',';

    while (get_line(line)) {
        printf("Line%d\n", count++);
        split(line, ret, sep, max);
        for (i = 0; i < max; i++) {
            printf(" >ret[%d] = '%s'\n", i, ret[i]);
        }
    }
}

\end{Verbatim}

%---------------------------------------------------------------%
\subsection{第3回 \texttt{parse\_line}関数} \label{code:3}
%---------------------------------------------------------------%

\begin{Verbatim}[numbers=left, xleftmargin=10mm, numbersep=6pt,
    fontsize=\small, baselinestretch=0.8]
#include <stdio.h>
#include <stdlib.h>
#include <string.h>

#define MAX_LINE_LEN 1024

void cmd_quit(void) {
    exit(0);
}

void cmd_check(char cmd) {
    fprintf(stderr, "%%%c command is invoked with no arg\n", cmd);
}

void cmd_print(char cmd, char *param) {
    fprintf(stderr, "%%%c command is invoked with arg: '%s'\n", cmd, param);
}

void cmd_read(char cmd, char *param) {
    fprintf(stderr, "%%%c command is invoked with arg: '%s'\n", cmd, param);
}

void cmd_write(char cmd, char *param) {
    fprintf(stderr, "%%%c command is invoked with arg: '%s'\n", cmd, param);
}

void cmd_find(char cmd, char *param) {
    fprintf(stderr, "%%%c command is invoked with arg: '%s'\n", cmd, param);
}

void cmd_sort(char cmd, char *param) {
    fprintf(stderr, "%%%c command is invoked with arg: '%s'\n", cmd, param);
}

void exec_command(char cmd, char *param) {
    switch (cmd) {
        case 'Q': cmd_quit();   break;
        case 'C': cmd_check(cmd);  break;
        case 'P': cmd_print(cmd, param);  break;
        case 'R': cmd_read(cmd, param);   break;
        case 'W': cmd_write(cmd, param);  break;
        case 'F': cmd_find(cmd, param);   break;
        case 'S': cmd_sort(cmd, param);   break;
        default:
        printf("Unregistered Command Is Entered.\n");
    }
}

int subst(char *str, char c1, char c2) {
    int diff = 0;
    char *p;

    p = str;
    while (*p != '\0') {
        if (*p == c1) {
            *p = c2;
            diff++;
        }
        p++;
    }
    return diff;
}

int split(char *str, char *ret[], char sep, int max) {
    int i, count = 0;
    subst(str, sep, '\0');
    for (i=0; i<max; i++) {
        ret[i] = str;
        str += strlen(str)+1;
        count++;
    }
    return count;
}

int get_line(char *line) {
    if (fgets(line, MAX_LINE_LEN + 1, stdin) == NULL || strlen(line) > MAX_LINE_LEN ||
                                                                        *line == '\n') {
        return 0;
    } else {
        subst(line, '\n', '\0');
        return 1;
    }
}

void new_profile(char *line) {
    int i, max = 5;
    char *ret[80] = {0}, sep = ',';
    split(line, ret, sep, max);
    for (i = 0; i < max; i++) {
        printf(" >ret[%d] = '%s'\n", i, ret[i]);
    }
}

void parse_line(char *line) {
    char cmd, param[80] = {0};
    if (*line == '%') {
        cmd = *(line+1);
        strcpy(param, line+3);
        exec_command(cmd, param);
    } else {
        new_profile(line);
    }
}

int main(void) {
    char line[MAX_LINE_LEN + 1];
    while (get_line(line)) {
        parse_line(line);
    }
    return 0;
}
\end{Verbatim}

%---------------------------------------------------------------%
\subsection{第4回 \texttt{new\_profile}関数} \label{code:4}
%---------------------------------------------------------------%

\begin{Verbatim}[numbers=left, xleftmargin=10mm, numbersep=6pt,
    fontsize=\small, baselinestretch=0.8]
#include <stdio.h>
#include <stdlib.h>
#include <string.h>

#define MAX_LINE_LEN 1024

struct date {
    int y;
    int m;
    int d;
};

struct profile {
    int id;
    char name[70];
    struct date birthday;
    char address[70];
    char note[80];
};

int profile_data_nitems = 0;
struct profile profile_data_store[10000];

void cmd_quit(void) {
    exit(0);
}

void cmd_check(char cmd) {
    fprintf(stderr, "%%%c command is invoked with no arg\n", cmd);
}

void cmd_print(char cmd, char *param) {
    fprintf(stderr, "%%%c command is invoked with arg: '%s'\n", cmd, param);
}

void cmd_read(char cmd, char *param) {
    fprintf(stderr, "%%%c command is invoked with arg: '%s'\n", cmd, param);
}

void cmd_write(char cmd, char *param) {
    fprintf(stderr, "%%%c command is invoked with arg: '%s'\n", cmd, param);
}

void cmd_find(char cmd, char *param) {
    fprintf(stderr, "%%%c command is invoked with arg: '%s'\n", cmd, param);
}

void cmd_sort(char cmd, char *param) {
    fprintf(stderr, "%%%c command is invoked with arg: '%s'\n", cmd, param);
}

void exec_command(char cmd, char *param) {
    switch (cmd) {
        case 'Q': cmd_quit();   break;
        case 'C': cmd_check(cmd);  break;
        case 'P': cmd_print(cmd, param);  break;
        case 'R': cmd_read(cmd, param);   break;
        case 'W': cmd_write(cmd, param);  break;
        case 'F': cmd_find(cmd, param);   break;
        case 'S': cmd_sort(cmd, param);   break;
        default:
        printf("Unregistered Command Is Entered.\n");
    }
}

int subst(char *str, char c1, char c2) {
    int diff = 0;
    char *p;

    p = str;
    while (*p != '\0') {
        if (*p == c1) {
            *p = c2;
            diff++;
        }
        p++;
    }
    return diff;
}

int split(char *str, char *ret[], char sep, int max) {
    int i, count = 0;
    subst(str, sep, '\0');
    for (i=0; i<max; i++) {
        ret[i] = str;
        str += strlen(str)+1;
        count++;
    }
    return count;
}

int get_line(char *line) {
    if (fgets(line, MAX_LINE_LEN + 1, stdin) == NULL || strlen(line) > MAX_LINE_LEN ||
                                                                        *line == '\n') {
        return 0;
    } else {
        subst(line, '\n', '\0');
        return 1;
    }
}

void new_profile(struct profile *profile_data_store, char *line) {
    int max_line = 5, max_date = 3;
    char *ret[80] = {0}, *date[80] = {0}, sep_line = ',', sep_date = '-';

    split(line, ret, sep_line, max_line);
    split(ret[2], date, sep_date, max_date);

    profile_data_store->id = atoi(ret[0]);
    strcpy(profile_data_store->name, ret[1]);

    profile_data_store->birthday.y = atoi(date[0]);
    profile_data_store->birthday.m = atoi(date[1]);
    profile_data_store->birthday.d = atoi(date[2]);

    strcpy(profile_data_store->address, ret[3]);
    strcpy(profile_data_store->note, ret[4]);
}

void parse_line(char *line) {
    char cmd, param[80] = {0};
    if (*line == '%') {
        cmd = *(line+1);
        strcpy(param, line+3);
        exec_command(cmd, param);
    } else {
        new_profile(&profile_data_store[profile_data_nitems++], line);
    }
}

int main(void) {
    char line[MAX_LINE_LEN + 1];
    while (get_line(line)) {
        parse_line(line);
    }
    return 0;
}

\end{Verbatim}

%---------------------------------------------------------------%
\subsection{第5回 \texttt{\%C \%P}コマンド} \label{code:5}
%---------------------------------------------------------------%

\begin{Verbatim}[numbers=left, xleftmargin=10mm, numbersep=6pt,
    fontsize=\small, baselinestretch=0.8]
#include <stdio.h>
#include <stdlib.h>
#include <string.h>

#define MAX_LINE_LEN 1024

struct date
{
    int y;
    int m;
    int d;
};

struct profile
{
    int id;
    char name[70];
    struct date birthday;
    char address[70];
    char note[80];
};

int profile_data_nitems = 0;
struct profile profile_data_store[10000];

void cmd_quit(void)
{
    exit(0);
}

void cmd_check(char cmd)
{
    printf("%d profile(s)\n", profile_data_nitems);
}

void cmd_print(char cmd, char *param)
{
    int count, num = atoi(param);
    if (num == 0) // 引数が0の場合
    {
        count = 0;
        while (count < profile_data_nitems)
        {
            printf("Id    : %d\n", profile_data_store[count].id);
            printf("Name  : %s\n", profile_data_store[count].name);
            printf("Birth : %04d-%02d-%02d\n", profile_data_store[count].birthday.y,
                profile_data_store[count].birthday.m, profile_data_store[count].birthday.d);
            printf("Addr. : %s\n", profile_data_store[count].address);
            printf("Comm. : %s\n\n", profile_data_store[count].note);
            count++;
        }
    }
    else if (num > 0) // 引数が正の場合
    {
        if (num > profile_data_nitems) // 要素数よりも大きい値が入力された場合
            num = profile_data_nitems;
        count = 0;
        while (count < num)
        {
            printf("Id    : %d\n", profile_data_store[count].id);
            printf("Name  : %s\n", profile_data_store[count].name);
            printf("Birth : %04d-%02d-%02d\n", profile_data_store[count].birthday.y,
                profile_data_store[count].birthday.m, profile_data_store[count].birthday.d);
            printf("Addr. : %s\n", profile_data_store[count].address);
            printf("Comm. : %s\n\n", profile_data_store[count].note);
            count++;
        }
    }
    else if (num < 0) // 引数が負の場合
    {
        if (num < -profile_data_nitems) // 要素数よりも大きい値が入力された場合
            num = -profile_data_nitems;
        count = profile_data_nitems + num;
        while (count < profile_data_nitems)
        {
            printf("Id    : %d\n", profile_data_store[count].id);
            printf("Name  : %s\n", profile_data_store[count].name);
            printf("Birth : %04d-%02d-%02d\n", profile_data_store[count].birthday.y,
                profile_data_store[count].birthday.m, profile_data_store[count].birthday.d);
            printf("Addr. : %s\n", profile_data_store[count].address);
            printf("Comm. : %s\n\n", profile_data_store[count].note);
            count++;
        }
    }
}

void cmd_read(char cmd, char *param)
{
    fprintf(stderr, "%%%c command is invoked with arg: '%s'\n", cmd, param);
}

void cmd_write(char cmd, char *param)
{
    fprintf(stderr, "%%%c command is invoked with arg: '%s'\n", cmd, param);
}

void cmd_find(char cmd, char *param)
{
    fprintf(stderr, "%%%c command is invoked with arg: '%s'\n", cmd, param);
}

void cmd_sort(char cmd, char *param)
{
    fprintf(stderr, "%%%c command is invoked with arg: '%s'\n", cmd, param);
}

void exec_command(char cmd, char *param)
{
    switch (cmd)
    {
    case 'Q':
        cmd_quit();
        break;
    case 'C':
        cmd_check(cmd);
        break;
    case 'P':
        cmd_print(cmd, param);
        break;
    case 'R':
        cmd_read(cmd, param);
        break;
    case 'W':
        cmd_write(cmd, param);
        break;
    case 'F':
        cmd_find(cmd, param);
        break;
    case 'S':
        cmd_sort(cmd, param);
        break;
    default:
        printf("Unregistered Command Is Entered.\n");
    }
}

int subst(char *str, char c1, char c2)
{
    int diff = 0;
    char *p;

    p = str; // 文字の先頭にポインタをあわせる
    while (*p != '\0')
    {
        if (*p == c1)
        {
            *p = c2;
            diff++;
        }
        p++;
    }
    return diff;
}

int split(char *str, char *ret[], char sep, int max)
{
    int i, count = 0;
    subst(str, sep, '\0'); // カンマをNULL終端に置き換え
    for (i = 0; i < max; i++)
    {
        ret[i] = str;
        str += strlen(str) + 1;
        count++;
    }
    return count;
}

int get_line(char *line)
{
    if (fgets(line, MAX_LINE_LEN + 1, stdin) == NULL || strlen(line) > MAX_LINE_LEN ||
                                                                            *line == '\n')
    {
        return 0;
    }
    else
    {
        subst(line, '\n', '\0');
        return 1;
    }
}

void new_profile(struct profile *profile_data_store, char *line)
{
    int max_line = 5, max_date = 3;
    char *ret[80] = {0}, *date[80] = {0}, sep_line = ',', sep_date = '-';

    split(line, ret, sep_line, max_line);
    split(ret[2], date, sep_date, max_date);

    profile_data_store->id = atoi(ret[0]);
    strcpy(profile_data_store->name, ret[1]);

    profile_data_store->birthday.y = atoi(date[0]);
    profile_data_store->birthday.m = atoi(date[1]);
    profile_data_store->birthday.d = atoi(date[2]);

    strcpy(profile_data_store->address, ret[3]);
    strcpy(profile_data_store->note, ret[4]);
}

void parse_line(char *line)
{
    char cmd, param[80] = {0};
    if (*line == '%')
    {
        cmd = *(line + 1);
        strcpy(param, line + 3);
        exec_command(cmd, param);
    }
    else
    {
        new_profile(&profile_data_store[profile_data_nitems++], line);
    }
}

int main(void)
{
    char line[MAX_LINE_LEN + 1];
    while (get_line(line))
    {
        parse_line(line);
    }
    return 0;
}

\end{Verbatim}

%---------------------------------------------------------------%
\subsection{第6回 最終コード} \label{code:6}
%---------------------------------------------------------------%

\begin{Verbatim}[numbers=left, xleftmargin=10mm, numbersep=6pt,
    fontsize=\small, baselinestretch=0.8]
#include <stdio.h>
#include <stdlib.h>
#include <string.h>

#define MAX_LINE_LEN 1024

struct date
{
    int y;
    int m;
    int d;
};

struct profile
{
    int id;
    char name[70];
    struct date birthday;
    char address[70];
    char note[1024];
};

int profile_data_nitems = 0;
struct profile profile_data_store[10000];

void cmd_quit(void)
{
    exit(0);
}

void cmd_check(char cmd)
{
    printf("%d profile(s)\n", profile_data_nitems);
}

void cmd_print(char cmd, char *param)
{
    int count, num = atoi(param);
    if (num == 0) // 引数が0の場合
    {
        count = 0;
        while (count < profile_data_nitems)
        {
            printf("Id    : %d\n", profile_data_store[count].id);
            printf("Name  : %s\n", profile_data_store[count].name);
            printf("Birth : %04d-%02d-%02d\n", profile_data_store[count].birthday.y,
                profile_data_store[count].birthday.m, profile_data_store[count].birthday.d);
            printf("Addr. : %s\n", profile_data_store[count].address);
            printf("Comm. : %s\n\n", profile_data_store[count].note);
            count++;
        }
    }
    else if (num > 0) // 引数が正の場合
    {
        if (num > profile_data_nitems) // 要素数よりも大きい値が入力された場合
            num = profile_data_nitems;
        count = 0;
        while (count < num)
        {
            printf("Id    : %d\n", profile_data_store[count].id);
            printf("Name  : %s\n", profile_data_store[count].name);
            printf("Birth : %04d-%02d-%02d\n", profile_data_store[count].birthday.y,
                profile_data_store[count].birthday.m, profile_data_store[count].birthday.d);
            printf("Addr. : %s\n", profile_data_store[count].address);
            printf("Comm. : %s\n\n", profile_data_store[count].note);
            count++;
        }
    }
    else if (num < 0) // 引数が負の場合
    {
        if (num < -profile_data_nitems) // 要素数よりも大きい値が入力された場合
            num = -profile_data_nitems;
        count = profile_data_nitems + num;
        while (count < profile_data_nitems)
        {
            printf("Id    : %d\n", profile_data_store[count].id);
            printf("Name  : %s\n", profile_data_store[count].name);
            printf("Birth : %04d-%02d-%02d\n", profile_data_store[count].birthday.y,
                profile_data_store[count].birthday.m, profile_data_store[count].birthday.d);
            printf("Addr. : %s\n", profile_data_store[count].address);
            printf("Comm. : %s\n\n", profile_data_store[count].note);
            count++;
        }
    }
}

void cmd_read(char cmd, char *param)
{
    fprintf(stderr, "%%%c command is invoked with arg: '%s'\n", cmd, param);
}

void cmd_write(char cmd, char *param)
{
    fprintf(stderr, "%%%c command is invoked with arg: '%s'\n", cmd, param);
}

void cmd_find(char cmd, char *param)
{
    fprintf(stderr, "%%%c command is invoked with arg: '%s'\n", cmd, param);
}

void cmd_sort(char cmd, char *param)
{
    fprintf(stderr, "%%%c command is invoked with arg: '%s'\n", cmd, param);
}

void exec_command(char cmd, char *param)
{
    switch (cmd)
    {
    case 'Q':
        cmd_quit();
        break;
    case 'C':
        cmd_check(cmd);
        break;
    case 'P':
        cmd_print(cmd, param);
        break;
    case 'R':
        cmd_read(cmd, param);
        break;
    case 'W':
        cmd_write(cmd, param);
        break;
    case 'F':
        cmd_find(cmd, param);
        break;
    case 'S':
        cmd_sort(cmd, param);
        break;
    default:
        break;
    }
}

int subst(char *str, char c1, char c2)
{
    int diff = 0;
    char *p;

    p = str; // 文字の先頭にポインタをあわせる
    while (*p != '\0')
    {
        if (*p == c1)
        {
            *p = c2;
            diff++;
        }
        p++;
    }
    return diff;
}

int split(char *str, char *ret[], char sep, int max)
{
    int count = 1;
    ret[0] = str;

    while (*str)
    {
        if (count >= max) // 規定値よりも多い時
            break;
        if (*str == sep)
        {
            *str = '\0'; // NULL終端に置き換え
            ret[count++] = str + 1;
        }
        str++;
    }

    return count;
}

int get_line(char *line)
{
    if (fgets(line, MAX_LINE_LEN + 1, stdin) == NULL || strlen(line) > MAX_LINE_LEN ||
                                                                            *line == '\n')
    {
        return 0;
    }
    else
    {
        subst(line, '\n', '\0');
        return 1;
    }
}

void new_profile(struct profile *profile_data_store, char *line)
{
    int max_line = 5, max_date = 3;
    char *ret[80] = {0}, *date[80] = {0}, sep_line = ',', sep_date = '-';

    split(line, ret, sep_line, max_line);
    split(ret[2], date, sep_date, max_date);

    profile_data_store->id = atoi(ret[0]);
    strcpy(profile_data_store->name, ret[1]);

    profile_data_store->birthday.y = atoi(date[0]);
    profile_data_store->birthday.m = atoi(date[1]);
    profile_data_store->birthday.d = atoi(date[2]);

    strcpy(profile_data_store->address, ret[3]);
    strcpy(profile_data_store->note, ret[4]);
}

void parse_line(char *line)
{
    char cmd, param[80] = {0};
    if (*line == '%')
    {
        cmd = *(line + 1);
        strcpy(param, line + 3);
        exec_command(cmd, param);
    }
    else
    {
        new_profile(&profile_data_store[profile_data_nitems++], line);
    }
}

int main(void)
{
    char line[MAX_LINE_LEN + 1];
    while (get_line(line))
    {
        parse_line(line);
    }
    return 0;
}

\end{Verbatim}

%---------------------------------------------------------------%
\subsection{\texttt{split}関数 (修正前)} \label{code:7}
%---------------------------------------------------------------%

\begin{Verbatim}[numbers=left, xleftmargin=10mm, numbersep=6pt,
    fontsize=\small, baselinestretch=0.8]
int split(char *str, char *ret[], char sep, int max)
{
    int i, count = 0;
    subst(str, sep, '\0'); // カンマをNULL終端に置き換え
    for (i = 0; i < max; i++)
    {
        ret[i] = str;
        str += strlen(str) + 1;
        count++;
    }
    return count;
}

\end{Verbatim}


%%%%%%%%%%%%%%%%%%%%%%%%%%%%%%%%%%%%%%%%%%%%%%%%%%%%%%%%%%%%%%%%%%%%%%
% 参考文献
%   実際に,参考にした書籍等の奥付などを見て書くこと.
%   本文で引用する際は,\cite{book:algodata}のように書けばよい.
%%%%%%%%%%%%%%%%%%%%%%%%%%%%%%%%%%%%%%%%%%%%%%%%%%%%%%%%%%%%%%%%%%%%%%
\begin{thebibliography}{99}
  \bibitem{book:algodata} 平田富雄,アルゴリズムとデータ構造,森北出版,1990.
  \bibitem{web:atoi} C言語のatoiで出来ること, https://arma-search.jp/article/clanguage-atoi, 2020/05/20.
\end{thebibliography}

%--------------------------------------------------------------------%
\end{document}
