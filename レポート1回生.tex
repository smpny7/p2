\documentclass[12pt]{jarticle}
\usepackage{amsmath}
\usepackage{amssymb}
\usepackage{ascmac}
\usepackage{graphicx}
\textwidth=16cm
\oddsidemargin=0cm

\newcommand \Br [1] {\left( #1 \right)}

% \renewcommand  \[  {\begin{eqnarray}}
% \renewcommand  \]  {\end{eqnarray}}

\begin{document}


\title{工学基礎実験実習 C言語の基礎 \\ | ニュートン法に関する実験 |}

\date{出題日: 2019年07月09日 \\
      提出日: 2019年07月23日 \\
      締切日: 2019年07月23日 \\}

\author{氏名: 池田 海斗 (IKEDA, Kaito) \\
        工学部 情報系学科 \\
        学生番号: 09501502}

\maketitle


%----------------------------------------------------------------------------------------------------%

\section{概要}

本レポートでは,以下の2つの課題について実験および考察を行う.

\begin{description}
\item 課題1
次の方程式の解をニュートン法を用いて求めよ.方程式を$f(x)=0$とおき,収束の様子を調べよ.また,収束の早さを調べ,グラフ等を用いてわかりやすく示せ.

\begin{enumerate}
\item $\sin e^x=0$
\item $x^3-3x-2=0$
\item 任意の方程式($x^3-3x^2-x+3=0$)
\end{enumerate}

\item 課題2
方程式$x^3-2x-5=0$の実数解を初期値$x_0=0$として,ニュートン法で求めよ.このとき,収束の様子を調べるため,$k$,$x_k$,$f(x_k)$,および$f^{\prime}(x_k)$の値を表示せよ.
\end{description}


%----------------------------------------------------------------------------------------------------%

\section{はじめに}

\begin{figure}[t]
\includegraphics[scale=0.75]{newtons.eps}
\caption{ニュートン法のグラフ}
\label{fig:newtons}
\end{figure}

\subsection{ニュートン法とは}

方程式の解を求める方法として,ニュートン法がある.ニュートン法とは,曲線を直線に近似して計算することで,$f(x)=0$の解$x$の近似を求めることが出来る計算法である.ニュートン法での導出手順は以下の通りである.

\begin{enumerate}
\item $y=f(x)$のグラフで,初期値$x_0$に適当な値を入れる.
\item $y=f(x)$のグラフで,$(x_0,f(x_0))$における接線を求める.
\item 求めた接線と$x$軸との交点$(x_1,0)$とする.
\item 同様に繰り返し$x$の値を更新していき,解を導出する.
\end{enumerate}

ここで,$(x_k,f(x_k))$における接線を求める式として,以下の式(\ref{eq:1})を使用する.
\[
y=f^{\prime}(x_k)(x-x_k)+f(x_k)
\label{eq:1}
\]
また,$x_{k+1}$の値を求める式として,以下の式(\ref{eq:2})を使用する.
\[
x_{k+1}=x_k-\frac{f(x_k)}{f^{\prime}(x_k)}
\label{eq:2}
\]

\subsection{ニュートン法の証明}

次に,上記のニュートン法が成り立つ証明をテイラー展開を用いて行う.
$f(x)=0$の収束値を$x^{\ast}$とすると,ニュートン法の第$k$ステップでの誤差$r_k$は,
\[
r_k&=&x_k-x^{\ast}
\label{eq:3}
\\
\therefore
r_{k+1}&=&x_{k+1}-x^{\ast}
\label{eq:4}
\]
とおける.$r_k$を用いて,$f(x)$を$x=x^{\ast}$の周りでテイラー展開すると,以下のようになる.
\[
f(x_k)&=&f(x^{\ast}+r_k) \nonumber \\
&=&f(x^{\ast})+f^{\prime}(x^{\ast})r_k+\frac{1}{2}f^{\prime\prime}(x^{\ast})r_k^2+O(r_k^3) \nonumber \\
&\approx&f^{\prime}(x^{\ast})r_k+\frac{1}{2}f^{\prime\prime}(x^{\ast})r_k^2
\label{eq:5}
\]

最終行については,$x^{\ast}$は$f(x)$の\emph{解であるから}$f(x^{\ast})=0$となり,また$r_k$は\emph{十分に小さいため},それを3乗した$r_k^3$についても近似して0とみなす.同様にして,$f^{\prime}(x)$もテイラー展開すると,
\[
f^{\prime}(x_k)&=&f^{\prime}(x^{\ast}+r_k) \nonumber \\
&=&f^{\prime}(x^{\ast})+f^{\prime\prime}(x^{\ast})r_k+\frac{1}{2}f^{\prime\prime\prime}(x^{\ast})r_k^2+O(r_k^3) \nonumber \\
&\approx&f^{\prime}(x^{\ast})+f^{\prime\prime}(x^{\ast})r_k
\label{eq:6}
\]

ここで,ニュートン法の反復式
\[
x_{k+1}=x_k-\frac{f(x_k)}{f^{\prime}(x_k)}
\label{eq:7}
\]
の両辺から$x^{\ast}$を引くと,式(\ref{eq:3})(\ref{eq:4})より,
\[
r_{k+1}=r_k-\frac{f(x_k)}{f^{\prime}(x_k)}
\label{eq:8}
\]
となり,これに式(\ref{eq:5})(\ref{eq:6})を代入すると,以下のようになる.
\[
r_{k+1}&=&r_k-\frac{f^{\prime}(x^{\ast})r_k+\frac{1}{2}f^{\prime\prime}(x_k)r_k^2}{f^{\prime}(x^{\ast})+f^{\prime\prime}(x^{\ast})r_k} \nonumber \\
&=&\frac{r_kf^{\prime}(x^{\ast})+r_kf^{\prime\prime}(x^{\ast})r_k^2-f^{\prime}(x^{\ast})r_k-\frac{1}{2}f^{\prime\prime}(x_k)r_k^2}{f^{\prime}(x^{\ast})+f^{\prime\prime}(x^{\ast})r_k}
\label{eq:9}
\]

ここで,$f^{\prime}(x^{\ast}) \gg f^{\prime\prime}(x^{\ast})r_k$より,また,今回の課題では\emph{重解を持たない方程式}であり,$f^{\prime}(x^{\ast})$が0になることはないため,$f^{\prime\prime}(x^{\ast})r_k$を0と近似し,消すことが出来る.よって,
\[
r_{k+1}=\frac{1}{2}\frac{f^{\prime\prime}(x^{\ast})}{f^{\prime}(x^{\ast})}r_k^2
\label{eq:10}
\]
となる.よって,ニュートン法は2次収束する解法である.


%----------------------------------------------------------------------------------------------------%

\section{プログラム及びその説明}
\subsection{ユーザ定義関数}

今回使うユーザ定義の関数は主に4つある.
\begin{enumerate}
\item \emph{f()}\qquad $x$の値を代入することで,$f(x)$の値を返す関数
\item \emph{f1()}\qquad $x$の値を代入することで,$f^{\prime}(x)$の値を返す関数
\item \emph{newton()}\qquad $x_k$の値を代入することで,$x_{k+1}$の値を返す関数
\item \emph{main()}\qquad 今回のプログラムの主な処理を行う関数
\end{enumerate}

また,ヘッダーファイルには,stdio.hとmath.hを使用している.

\subsection{使用した変数}

\begin{enumerate}
\item \emph{n}\qquad 繰り返し回数を記録する.
\item \emph{ans}\qquad 解を格納しておき,終了条件に使用する.
\item \emph{delta}\qquad 許容誤差値を格納しておき,解との差を使って終了条件に使用する.
\item \emph{xk}\qquad k回目の$x$の値である$x_k$を記録する.
\end{enumerate}

\subsection{終了条件}

求めた値\emph{xk}と実際の解\emph{ans}との差の絶対値が,許容誤差値\emph{delta}よりも小さければ終了する.

\subsection{実際に使用したプログラム}

実際に使用したプログラムのコードは,レポートの最後に掲載する.(図\ref{fig:p1a}, \ref{fig:p1b}, \ref{fig:p1c}, \ref{fig:p2})


%----------------------------------------------------------------------------------------------------%

\section{プログラムの使用法}

実際にプログラムを動かす際には,以下の三点を行う必要がある.
\begin{itemize}
\item 求めたい方程式の解を求め,許容誤差値とともに入力する.
\item 求める方程式,その微分した式を関数内に記載する.
\item 初期値,終了条件について考察をし,適切かどうかを判断して訂正する.
\end{itemize}

\subsection{課題1-(a)}
課題1-(a)について,解は$\log\pi$である.今回解が$1.14\cdots$であるので,初期値を$1$に設定し,許容誤差は$3.0 \times 10^{-16}$とする.グラフと実行結果については図\ref{fig:1a}, \ref{fig:1aa}に記載する.

ここで,収束の速さについても調査する.収束の速さは図\ref{fig:1a+a}の通りであった.

\subsection{課題1-(b)}
課題1-(b)について,解は$-1$, $2$である.しかし$-1$は重解であり,\emph{上記のニュートン法の証明の前提条件}として,解で$f'(x)\neq0$としているため,今回は解を$2$とする.初期値を$3$とし,許容誤差は同様に$3.0 \times 10^{-16}$とする.グラフと実行結果については図\ref{fig:1b}, \ref{fig:1bb}に記載する.また,収束の速さについては図\ref{fig:1b+a}の通りであった.

\subsection{課題1-(c)}
課題1-(c)について,解は$-1$, $1$, $3$である.今回解を$3$とし,解に近づけるよう初期値を$5$に設定した.他の値に設定しても動作するようにしたいので,\emph{後ほど考察にて深く調査していきたい}.同様に許容誤差を$3.0 \times 10^{-16}$とし,グラフと収束の速さ,実行結果については図\ref{fig:1c}, \ref{fig:1c+a}, \ref{fig:1cc}に記載する.

\subsection{課題2}
課題2について,解は$2.09\cdots$となった.課題1と同様に許容誤差を$3.0 \times 10^{-16}$とすると,いつまで経っても解が求まらなかった.そこで,許容誤差の値をもう少し大きくしてみると,解が一定値に収まった.この可否の境界についても,\emph{後ほど考察にて調査したい}.グラフと実行結果は図\ref{fig:2}, \ref{fig:22}に記載する.


%----------------------------------------------------------------------------------------------------%

\section{プログラムの考察}

疑問を持ったところやプログラムの改善すべき点について,ここでは追求していきたい.
上記に記載した通り,ここでは以下の二点について深層調査を行う.

\begin{enumerate}
\item 課題1の(c)など,解が複数ある方程式において,すべての解を求められるようにしたい.
\item 課題2の解が求まるギリギリの許容誤差の値はいくらなのか.
\end{enumerate}

\subsection{すべての解を求めるプログラム}
課題1の(a)のプログラムでは,解を一つしか求めることができなかった.勿論,重解の場合などはニュートン法では求められない.課題1の発展として,解をすべて求められるプログラムを作りたい.

図\ref{fig:3}のように,細分化してたくさんの区間を作り,$f(x_k) \times f(x_{k+1}) \leqq 0$となる所を探索していくプログラムである.この区間の間隔は,とりあえず$1.0 \times 10^{-3}$とする.この探索は$0$から正に向かって探索を行い,次に負を行う.範囲は絶対値$10$の間とする.プログラムは図\ref{fig:3a}に記載する.

\begin{itemize}
\item 課題1(c)の解は,$-1$, $1$, $3$であり,このプログラムで実行してみた.(図\ref{fig:3b})
\item 課題1(a)の$\sin e^x=0$についても,きちんと解が出ることが確認できた.(図\ref{fig:3c})
\end{itemize}

プログラムを見ると分かると思うが,$f^{\prime}(x_k)$の積についても条件分岐をしている.今回の式にはあまり意味は無いが,$f^{\prime}(x_k) \times f^{\prime}(x_{k+1}) \leqq 0$となる場合で条件分岐することで,重解の場合も対処できると考えた.時間が足りず実装できなかったが,時間がある時にチャレンジしてみたい.

\subsection{許容誤差の値選定}

課題2について,許容誤差の値を$1.0 \times 10^{-15}$にすると上手くいったが,$1.0 \times 10^{-16}$にすると上手くいかなかった.誰しもが,その境界について気になったであろう.\ldots 少なくとも私は気になったので,探求していきたいと思う.

プログラムの概要としては,$1.0 \times 10^{-15}$から$1.0 \times 10^{-16}$の間を,小数第16位から17,18\ldots と順に値を当てはめていく.もし30回以内に収束しなかった場合,一つ前の数字に戻り,位の1つ小さい所の探索を行う.double型とはいえ,やはり位が小さすぎると演算誤差が出るので,小数第30位で丸めることとする.

プログラムは図\ref{fig:4a}に,実行結果は図\ref{fig:4b}に記載する.実行結果より,$8.88178419700126 \times 10^{-16}$ (小数第30位まで) であることが分かった.


%----------------------------------------------------------------------------------------------------%

\section{まとめ}

今回のレポートでは,方程式の解をニュートン法で求めるプログラムを作成した.ニュートン法の証明についても右田先生にご教授いただき,重解の場合などに式が成り立たないことも理解できた.またオリジナルのプログラムを2つ作成することで,疑問に思っていたところを深く追求していくことができた.

それと同時に,まだまだ調査を行いたいところが出てきた.例えば,考察で述べた方法で重解を持つ場合に場合分けを行い,課題1(b)でも動作するようにしたり,その$1.0 \times 10^{-3}$単位区間においても,考察1と考察2のプログラムを組み合わせ,単位区間内でも同様に探索プログラムを実行させ,小数第30位まで正確に解を求められるようなプログラムも作成したかった.

また,今回30桁以下では桁落ちが発生していたが,プログラムを作成し終えた後でlong double型というものがあることを知った.この変数を使っていれば,もっと精密な値が出ていたかもしれない.
















































%----------------------------------------------------------------------------------------------------%

\begin{figure}[t]
 \begin{minipage}{7.95cm}
  \center
  \includegraphics[scale=.35]{graph1a.eps}
  \caption{$f(x)=\sin e^x$}
  \label{fig:1a}
 \end{minipage}
 \begin{minipage}{7.95cm}
  \center
  \includegraphics[scale=.35]{graph1a+a.eps}
  \caption{収束の様子 1 - (a)}
  \label{fig:1a+a}
 \end{minipage}
\begin{screen}
\begin{verbatim}
     1	Newton method program start.
     2	n:1 xk: 1.165748 f(xk):-0.066679 ans-xk:-0.021018
     3	n:2 xk: 1.144918 f(xk):-0.000592 ans-xk:-0.000188
     4	n:3 xk: 1.144730 f(xk):-0.000000 ans-xk:-0.000000
     5	n:4 xk: 1.144730 f(xk):-0.000000 ans-xk:-0.000000
     6	n:5 xk: 1.144730 f(xk): 0.000000 ans-xk: 0.000000
     7	done.
\end{verbatim}
\end{screen}
\caption{実行結果 1 - (a)}
\label{fig:1aa}
\end{figure}

\begin{figure}[t]
 \begin{minipage}{7.95cm}
  \center
  \includegraphics[scale=.35]{graph1b.eps}
  \caption{$f(x)=x^3-3x-2$}
  \label{fig:1b}
 \end{minipage}
 \begin{minipage}{7.95cm}
  \center
  \includegraphics[scale=.35]{graph1b+a.eps}
  \caption{収束の様子 1 - (b)}
  \label{fig:1b+a}
 \end{minipage}
\begin{screen}
\begin{verbatim}
     1	Newton method program start.
     2	n:1 xk: 2.333333 f(xk): 3.703704 ans-xk:-0.333333
     3	n:2 xk: 2.055556 f(xk): 0.518690 ans-xk:-0.055556
     4	n:3 xk: 2.001949 f(xk): 0.017567 ans-xk:-0.001949
     5	n:4 xk: 2.000003 f(xk): 0.000023 ans-xk:-0.000003
     6	n:5 xk: 2.000000 f(xk): 0.000000 ans-xk:-0.000000
     7	n:6 xk: 2.000000 f(xk): 0.000000 ans-xk: 0.000000
     8	done.
\end{verbatim}
\end{screen}
\caption{実行結果 1 - (b)}
\label{fig:1bb}
\end{figure}

\begin{figure}[t]
 \begin{minipage}{7.95cm}
  \center
  \includegraphics[scale=.35]{graph1c.eps}
  \caption{$f(x)=x^3-3x^2-x+3$}
  \label{fig:1c}
 \end{minipage}
 \begin{minipage}{7.95cm}
  \center
  \includegraphics[scale=.35]{graph1c+a.eps}
  \caption{収束の様子 1 - (c)}
  \label{fig:1c+a}
 \end{minipage}
\begin{screen}
\begin{verbatim}
     1	Newton method program start.
     2	n:1 xk:3.909091 f(xk):12.982720 ans-xk:-0.909091
     3	n:2 xk:3.302094 f(xk):2.991881 ans-xk:-0.302094
     4	n:3 xk:3.050652 f(xk):0.420738 ans-xk:-0.050652
     5	n:4 xk:3.001817 f(xk):0.014555 ans-xk:-0.001817
     6	n:5 xk:3.000002 f(xk):0.000020 ans-xk:-0.000002
     7	n:6 xk:3.000000 f(xk):0.000000 ans-xk:-0.000000
     8	n:7 xk:3.000000 f(xk):0.000000 ans-xk:0.000000
     9	done.
\end{verbatim}
\end{screen}
\caption{実行結果 1 - (c)}
\label{fig:1cc}
\end{figure}

\begin{figure}[t]
  \center
  \includegraphics[scale=.35]{graph2.eps}
  \caption{$f(x)=x^3-3x^2-x+3$}
  \label{fig:2}
\begin{screen}
\begin{verbatim}
     1	Newton method program start.
     2	k: 1 xk:-2.500000 f(xk):-15.625000 f'(xk):16.750000
     3	k: 2 xk:-1.567164 f(xk): -5.714632 f'(xk): 5.368011
     4	k: 3 xk:-0.502592 f(xk): -4.121770 f'(xk):-1.242203
     5	k: 4 xk:-3.820706 f(xk):-53.132488 f'(xk):41.793394
     6	k: 5 xk:-2.549393 f(xk):-16.470758 f'(xk):17.498220
     7	k: 6 xk:-1.608111 f(xk): -5.942390 f'(xk): 5.758068
     8	k: 7 xk:-0.576100 f(xk): -4.039002 f'(xk):-1.004325
     9	k: 8 xk:-4.597710 f(xk):-92.995258 f'(xk):61.416800
    10	k: 9 xk:-3.083543 f(xk):-28.151977 f'(xk):26.524715
    11	k:10 xk:-2.022194 f(xk): -9.224909 f'(xk):10.267809
    12	k:11 xk:-1.123764 f(xk): -4.171613 f'(xk): 1.788537
    13	k:12 xk: 1.208652 f(xk): -5.651658 f'(xk): 2.382516
    14	k:13 xk: 3.580790 f(xk): 33.751515 f'(xk):36.466172
    15	k:14 xk: 2.655233 f(xk):  8.409627 f'(xk):19.150790
    16	k:15 xk: 2.216106 f(xk):  1.451367 f'(xk):12.733381
    17	k:16 xk: 2.102125 f(xk):  0.084892 f'(xk):11.256789
    18	k:17 xk: 2.094584 f(xk):  0.000358 f'(xk):11.161841
    19	k:18 xk: 2.094551 f(xk):  0.000000 f'(xk):11.161438
    20	k:19 xk: 2.094551 f(xk): -0.000000 f'(xk):11.161438
    21	done.
\end{verbatim}
\end{screen}
\caption{実行結果 2}
\label{fig:22}
\end{figure}

\begin{figure}[t]
  \center
  \includegraphics[scale=.6]{graph3.eps}
  \caption{考察1 解説}
  \label{fig:3}
\end{figure}
\begin{figure}[t]
\begin{screen}
\begin{verbatim}
     1	解は,1.00 3.00 -1.00 です.
\end{verbatim}
\end{screen}
\caption{考察1 $x^3-3x^2-x+3=0$での実行結果}
\label{fig:3b}
\begin{screen}
\footnotesize
\begin{verbatim}
  1  解は,1.14 1.84 2.24 2.53 2.75 2.94 3.09 3.22 3.34 3.45 3.54 3.63 3.71 3.78
    3.85 3.92 3.98 4.03 4.09 4.14 4.19 4.23 4.28 4.32 4.36 4.40 4.44 4.48 4.51
    4.54 4.58 4.61 4.64 4.67 4.70 4.73 4.75 4.78 4.81 4.83 4.86 4.88 4.90 4.93
    4.95 4.97 4.99 5.02 5.04 5.06 5.08 5.10 5.12 5.13 5.15 5.17 5.19 5.21 5.22
    5.24 5.26 5.27 5.29 5.30 5.32 5.33 5.35 5.36 5.38 5.39 5.41 5.42 5.44 5.45
    5.46 5.48 5.49 5.50 5.51 5.53 5.54 5.55 5.56 5.58 5.59 5.60 5.61 5.62 5.63
    5.64 5.66 5.67 5.68 5.69 5.70 5.71 5.72 5.73 5.74 5.75 です.
\end{verbatim}
\end{screen}
\caption{考察1 $\sin e^x=0$での実行結果}
\label{fig:3c}
\end{figure}

\begin{figure}[t]
\begin{screen}
\small
\begin{verbatim}
     1	Newton method program start.
     2	0.000000000000001000000000000000の時は収束しました.
     3	0.000000000000000900000000000000の時は収束しました.
     4	0.000000000000000800000000000000の時は収束しませんでした.
     5	0.000000000000000890000000000000の時は収束しました.
     6	0.000000000000000880000000000000の時は収束しませんでした.
     7	0.000000000000000889000000000000の時は収束しました.
     8	0.000000000000000888000000000000の時は収束しませんでした.
     9	0.000000000000000888900000000000の時は収束しました.
    10	0.000000000000000888800000000000の時は収束しました.
    11	0.000000000000000888700000000000の時は収束しました.
    12	0.000000000000000888600000000000の時は収束しました.
    13	0.000000000000000888500000000000の時は収束しました.
    14	0.000000000000000888400000000000の時は収束しました.
    15	0.000000000000000888300000000000の時は収束しました.
    16	0.000000000000000888200000000001の時は収束しました.
    17	0.000000000000000888100000000000の時は収束しませんでした.
    18	0.000000000000000888190000000001の時は収束しました.
    19	0.000000000000000888180000000001の時は収束しました.
    20	0.000000000000000888170000000001の時は収束しませんでした.
    21	0.000000000000000888179000000001の時は収束しました.
    22	0.000000000000000888178000000001の時は収束しませんでした.
    23	0.000000000000000888178900000001の時は収束しました.
    24	0.000000000000000888178800000001の時は収束しました.
    25	0.000000000000000888178700000001の時は収束しました.
    26	0.000000000000000888178600000001の時は収束しました.
    27	0.000000000000000888178500000001の時は収束しました.
    28	0.000000000000000888178400000001の時は収束しませんでした.
    29	0.000000000000000888178490000001の時は収束しました.
    30	0.000000000000000888178480000001の時は収束しました.
    31	0.000000000000000888178470000001の時は収束しました.
    32	0.000000000000000888178460000001の時は収束しました.
    33	0.000000000000000888178450000001の時は収束しました.
    34	0.000000000000000888178440000001の時は収束しました.
    35	0.000000000000000888178430000001の時は収束しました.
    36	0.000000000000000888178420000001の時は収束しました.
    37	0.000000000000000888178410000001の時は収束しませんでした.
    38	0.000000000000000888178419000001の時は収束しませんでした.
    39	0.000000000000000888178419900001の時は収束しました.
    40	0.000000000000000888178419800001の時は収束しました.
    41	0.000000000000000888178419700001の時は収束しませんでした.
    42	0.000000000000000888178419790001の時は収束しました.
    43	0.000000000000000888178419780001の時は収束しました.
    44	0.000000000000000888178419770001の時は収束しました.
    45	0.000000000000000888178419760001の時は収束しました.
\end{verbatim}
\end{screen}
\end{figure}
\begin{figure}[t]
\begin{screen}
\small
\begin{verbatim}
    46	0.000000000000000888178419750001の時は収束しました.
    47	0.000000000000000888178419740001の時は収束しました.
    48	0.000000000000000888178419730001の時は収束しました.
    49	0.000000000000000888178419720001の時は収束しました.
    50	0.000000000000000888178419710001の時は収束しました.
    51	0.000000000000000888178419700001の時は収束しませんでした.
    52	0.000000000000000888178419709001の時は収束しました.
    53	0.000000000000000888178419708001の時は収束しました.
    54	0.000000000000000888178419707001の時は収束しました.
    55	0.000000000000000888178419706000の時は収束しました.
    56	0.000000000000000888178419705000の時は収束しました.
    57	0.000000000000000888178419704000の時は収束しました.
    58	0.000000000000000888178419703000の時は収束しました.
    59	0.000000000000000888178419702000の時は収束しました.
    60	0.000000000000000888178419701000の時は収束しました.
    61	0.000000000000000888178419700000の時は収束しませんでした.
    62	0.000000000000000888178419700900の時は収束しました.
    63	0.000000000000000888178419700800の時は収束しました.
    64	0.000000000000000888178419700700の時は収束しました.
    65	0.000000000000000888178419700600の時は収束しました.
    66	0.000000000000000888178419700500の時は収束しました.
    67	0.000000000000000888178419700400の時は収束しました.
    68	0.000000000000000888178419700300の時は収束しました.
    69	0.000000000000000888178419700200の時は収束しました.
    70	0.000000000000000888178419700100の時は収束しませんでした.
    71	0.000000000000000888178419700190の時は収束しました.
    72	0.000000000000000888178419700180の時は収束しました.
    73	0.000000000000000888178419700170の時は収束しました.
    74	0.000000000000000888178419700160の時は収束しました.
    75	0.000000000000000888178419700150の時は収束しました.
    76	0.000000000000000888178419700140の時は収束しました.
    77	0.000000000000000888178419700130の時は収束しました.
    78	0.000000000000000888178419700120の時は収束しませんでした.
    79	0.000000000000000888178419700129の時は収束しました.
    80	0.000000000000000888178419700128の時は収束しました.
    81	0.000000000000000888178419700127の時は収束しました.
    82	0.000000000000000888178419700126の時は収束しました.
    83	0.000000000000000888178419700125の時は収束しませんでした.
    84	0.000000000000000888178419700126の時は収束しました.
    85	done.
\end{verbatim}
\end{screen}
\caption{考察2 実行結果}
\label{fig:4b}
\end{figure}



%----------------------------------------------------------------------------------------------------%

\begin{figure}[t]
\begin{screen}
\footnotesize
\begin{verbatim}
     1	#include <stdio.h>
     2	#include <math.h> 
     3	
     4	double f(double x)
     5	{
     6	    return sin(exp(x));
     7	}
     8	
     9	double f1(double x)
    10	{
    11	    return exp(x) * cos(exp(x));
    12	}
    13	
    14	double newton(double xk)
    15	{
    16	    return xk - (f(xk) / f1(xk));
    17	}
    18	
    19	main()
    20	{
    21	    int n = 0;
    22	    double ans = log(3.14159265358979323846);
    23	    double delta = 3E-16;
    24	
    25	    double xk = 1;
    26	
    27	    printf("Newton method program start.\n");
    28	
    29	    while (fabs(f(xk)) > delta){
    30	        n = n + 1;
    31	        xk = newton(xk);
    32	        printf("n:%d xk:%f f(xk):%f ans-xk:%f\n", n, xk, f(xk), ans - xk);
    33	    }
    34	    printf("done.\n");
    35	}
\end{verbatim}
\end{screen}
\caption{課題1(a) プログラム}
\label{fig:p1a}
\end{figure}

\begin{figure}[t]
\begin{screen}
\footnotesize
\begin{verbatim}
     1	#include <stdio.h>
     2	#include <math.h> 
     3	
     4	double f(double x)
     5	{
     6	    return x*x*x-3*x-2;
     7	}
     8	
     9	double f1(double x)
    10	{
    11	    return 3*x*x-3;
    12	}
    13	
    14	double newton(double xk)
    15	{
    16	    return xk - (f(xk) / f1(xk));
    17	}
    18	
    19	main()
    20	{
    21	    int n = 0;
    22	    double ans = 2;
    23	    double delta = 3E-16;
    24	
    25	    double xk = 3;
    26	
    27	    printf("Newton method program start.\n");
    28	
    29	    while (fabs(f(xk)) > delta){
    30	        n = n + 1;
    31	        xk = newton(xk);
    32	        printf("n:%d xk:%f f(xk):%f ans-xk:%f\n", n, xk, f(xk), ans - xk);
    33	    }
    34	    printf("done.\n");
    35	}
\end{verbatim}
\end{screen}
\caption{課題1(b) プログラム}
\label{fig:p1b}
\end{figure}

\begin{figure}[t]
\begin{screen}
\footnotesize
\begin{verbatim}
     1	#include <stdio.h>
     2	#include <math.h> 
     3	
     4	double f(double x)
     5	{
     6	    return x*x*x-3*x*x-x+3;
     7	}
     8	
     9	double f1(double x)
    10	{
    11	    return 3*x*x-6*x-1;
    12	}
    13	
    14	double newton(double xk)
    15	{
    16	    return xk - (f(xk) / f1(xk));
    17	}
    18	
    19	main()
    20	{
    21	    int n = 0;
    22	    double ans = 3;
    23	    double delta = 3E-16;
    24	
    25	    double xk = 5;
    26	
    27	    printf("Newton method program start.\n");
    28	
    29	    while (fabs(f(xk)) > delta){
    30	        n = n + 1;
    31	        xk = newton(xk);
    32	        printf("n:%d xk:%f f(xk):%f ans-xk:%f\n", n, xk, f(xk), ans - xk);
    33	    }
    34	    printf("done.\n");
    35	}
\end{verbatim}
\end{screen}
\caption{課題1(c) プログラム}
\label{fig:p1c}
\end{figure}

\begin{figure}[t]
\begin{screen}
\footnotesize
\begin{verbatim}
     1	#include <stdio.h>
     2	#include <math.h> 
     3	
     4	double f(double x)
     5	{
     6	    return x*x*x-2*x-5;
     7	}
     8	
     9	double f1(double x)
    10	{
    11	    return 3*x*x-2;
    12	}
    13	
    14	double newton(double xk)
    15	{
    16	    return xk - (f(xk) / f1(xk));
    17	}
    18	
    19	main()
    20	{
    21	    int k = 0;
    22	    double delta = 1E-15;
    23	
    24	    double xk = 0;
    25	
    26	    printf("Newton method program start.\n");
    27	
    28	    while (fabs(f(xk)) > delta){
    29	        k = k + 1;
    30	        xk = newton(xk);
    31	        printf("k:%2d xk:%2.6f f(xk):%2f f'(xk):%2f\n", k, xk, f(xk), f1(xk));
    32	    }
    33	    printf("done.\n");
    34	}
\end{verbatim}
\end{screen}
\caption{課題2 プログラム}
\label{fig:p2}
\end{figure}

\begin{figure}[t]
\begin{screen}
\footnotesize
\begin{verbatim}
     1	#include <stdio.h>
     2	#include <math.h> 
     3	
     4	double f(double x)
     5	{
     6	    return x*x*x-3*x*x-x+3;
     7	}
     8	
     9	double f1(double x)
    10	{
    11	    return 3*x*x-6*x-1;
    12	}
    13	
    14	double newton(double xk)
    15	{
    16	    return xk - (f(xk) / f1(xk));
    17	}
    18	
    19	main()
    20	{
    21	
    22	    double delta = 1E-3; /* 間隔 */
    23	    double x = 0; /* 探索の現在地 */
    24	    int a = 0; /* 配列の要素番号 */
    25	    int i; /* printfに使う変数 */
    26	    double array[99]; /* 配列(Max99) */
    27	
    28	    while (fabs(x) <= 10){
    29	        if (f(x + delta) * f(x) <= 0 && f1(x + delta) * f1(x) >= 0){
    30	            array[a] = x;
    31	            a = a + 1;
    32	        }
    33	        x = x + delta;
    34	    }
    35	
    36	    x = 0;
    37	
    38	    while (fabs(x) <= 10){
    39	        if (f(x + delta) * f(x) <= 0 && f1(x + delta) * f1(x) >= 0){
    40	            array[a] = x;
    41	            a = a + 1;
    42	        }
    43	        x = x - delta;
    44	    }
    45	
    46	    printf("解は,");
    47	    for (i = 0; i <= a-1; ++i){
    48	        printf("%.2f ",array[i]);
    49	    }
    50	    printf("です.\n");
    51	    
    52	}
\end{verbatim}
\end{screen}
\caption{考察1 プログラム}
\label{fig:3a}
\end{figure}

\begin{figure}[t]
\begin{screen}
\footnotesize
\begin{verbatim}
     1	#include <stdio.h>
     2	#include <math.h> 
     3	
     4	double f(double x)
     5	{
     6	    return x*x*x-2*x-5;
     7	}
     8	
     9	double f1(double x)
    10	{
    11	    return 3*x*x-2;
    12	}
    13	
    14	double newton(double xk)
    15	{
    16	    return xk - (f(xk) / f1(xk));
    17	}
    18	
    19	main()
    20	{
    21	    printf("Newton method program start.\n");
    22	    double delta = 1E-15;
    23	    double delta2 = 1E-16;
    24	    int a = 0;
    25	
    26	    while (a < 3){
    27	
    28	        int n = 0;
    29	
    30	        while (n < 10){
    31	            int k = 0;
    32	            double xk = 0;
    33	
    34	            while (fabs(f(xk)) > delta && k <50){
    35	                k = k + 1;
    36	                xk = newton(xk);
    37	            }
    38	            if (k != 50){
    39	                printf("%2.30fの時は収束しました.\n", delta);
    40	                delta = delta - delta2;
    41	            }else{
    42	                printf("%2.30fの時は収束しませんでした.\n", delta);
    43	                delta2 = delta2 / 10;
    44	                delta = delta2 * 9 + delta;
    45	                a = a + 1;
    46	                n = -1;
    47	            }
    48	            n = n + 1;
    49	        }
    50	    }
    51	    printf("done.\n");
    52	}
\end{verbatim}
\end{screen}
\caption{考察2 プログラム}
\label{fig:4a}
\end{figure}


%----------------------------------------------------------------------------------------------------%

\thebibliography{99}
 \bibitem{literature1} http://www.swlab.cs.okayama-u.ac.jp/~gotoh/lect/p1/c\_prog/c3/newton\_proof.pdf
 \bibitem{literaturel} http://www.swlab.cs.okayama-u.ac.jp/~gotoh/lect/p1/c\_prog/c3/newton\_repetition.pdf

\end{document}
